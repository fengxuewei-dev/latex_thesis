\chapter{软件开发}
    \section{需求分析}
        \subsection{引言}
          \subsubsection{编写目的}
            本需求的编写目的在于研究无人机集群编队任务的开发途径和应用方法, 为以后的开发工作提供可靠的依据. 
          \subsubsection{项目背景}
            本课题的研究开发依赖于ROS平台, PX4开源飞控代码, QGC, 以及软件仿真平台(Gazebo), 硬件仿真平台(XPlane). 各自对应的关系如图\ref{simulator}所示. 
            \begin{figure}[htbp]
              \centering
              \subfigure[Software In the Loop]{
                  \begin{minipage}[t]{0.48\linewidth}
                  \centering
                  \includegraphics[width=0.9\textwidth]{pictures/sitl.png}
                  % \caption{Software In the Loop}
                  \end{minipage}%
              }%
              \subfigure[Software In the Loop]{
                \begin{minipage}[t]{0.48\linewidth}
                      \centering
                      \includegraphics[width=0.9\textwidth]{pictures/hitl.png}
                      % \caption{Hardware In the Loop}
                  \end{minipage}%
              }%
              \caption{仿真平台}
              \label{simulator}
            \end{figure}
          
          \subsubsection{数据字典}
          \begin{itemize}
            \item [(1)] QGC$\_$COMMAND指令: 
            \item 用处: 集群地面站为每一架飞机发送的指令
            \item 组成: 飞机的id编号 + 飞机分组之后的组编号 + 主从机标志位 + 任务类型 + 其他的一些数据
          \end{itemize}
      
        \subsection{功能需求}
            把系统主要功能包括实时子系统, 导航子系统, 制导子系统, 姿态子系统, 通信子系统, 主控, 以及飞行日志记录. 
            其中实时子系统主要负责和PX4进行交互, 获取当前系统的实时位姿数据, 供其他子系统进行使用; 
            导航子系统包含了路径管理和期望航点的获取, 将其下发给制导子系统; 
            制导子系统包含了直线控制逻辑和盘旋控制逻辑, 并依托于控制逻辑计算期望姿态数据, 最后下发给姿态子系统; 
            姿态子系统负责将制导子系统获得的期望姿态进行PID控制,最终以某种方式将控制量发送给PX4;
            通信子系统在集群系统数据流下担任了主要角色, 它是无人机群组之间, 无人机相互之间, 无人机与集群地面站之间的沟通桥梁; 
            飞行日志记录主要担任着飞行数据的保存. 
            \par
            系统功能图如下: 
            \begin{figure}[htbp]
              \centering
              \includegraphics[width=0.5\textwidth]{pictures/single_system.png}
              \label{single_system}
            \end{figure}
    \clearpage
    \section{总体设计}
      \begin{figure}[h]
        \centering
        \includegraphics[width=\textwidth]{pictures/modules.png}
        \caption{UAV逻辑图}
        \label{fig:UAV}
      \end{figure}
        \subsection{处理流程}
        集群编队, 当所有飞机全部上电起飞后(\ref{fig:groups}), 等待集群地面站为每一架飞机发送指令(QGC$\_$COMMNAD), 对应无人机接收到集群地面站发送给自己的指令, 进行解析, 指定无人机当前的执行任务以及一个所属属性. \par
        若指定当前无人机为主机, 那么主机执行特定算法, 计算期望姿态组, 按照某种特定方式发布给PX4, 进而交给PX4的执行器进行执行, 同时广播自己特定的位姿数据到同组的无人机, 为后序的编队或者自组织服务, 数据流图见\ref{fig:leader}; 
        若指定当前无人机为从机, 那么接收其他无人机广播过来的位姿信息, 筛选出同组主机位姿信息, 进行计算, 完成特定任务, 见数据流图\ref{fig:follower}.
        
        \begin{figure}
          \centering
          \includegraphics[width=\textwidth]{pictures/groups.pdf}
          \caption{组间分布图}
          \label{fig:groups}
        \end{figure}
        \begin{figure}
          \centering
          \includegraphics[width=\textwidth]{pictures/leader.png}
          \caption{主机数据流图}
          \label{fig:leader}
        \end{figure}
        \begin{figure}
          \centering
          \includegraphics[width=\textwidth]{pictures/follower.pdf}
          \caption{从机数据流图}
          \label{fig:follower}
        \end{figure}

        \subsection{总体结构和模块外部设计}
        因每一架飞机根据任务分配的不同, 所属于的组名, 以及在组内所担任的角色也是变化的, 所以需要将主机和从机的两套执行控制逻辑整合成为一套, 在内部根据飞机控制模式进行判断选择所要执行的逻辑. 所以当前无人机的执行控制逻辑如图\ref{fig:UAV}所示. 
        
        
        \clearpage
    \section{详细设计}
      \subsection{实时系统}
    实时系统担任的主要角色是和PX4进行交互. 如图\ref{fig:msgsubandpub}所示, 实时系统从px4中读取到当前系统的实时位姿信息之后, 将其分别下发到下面几个控制器进行一些逻辑方面的计算. 
    \begin{figure}[htbp]
        \centering
        \includegraphics[width=0.9\textwidth]{pictures/system_2.png}
        \caption{实时子系统}
    \label{fig:msgsubandpub}
  \end{figure}
\subsection{主控模块}
    主控模块在系统中担任着主要控制的作用, 任何模块之间的相互控制, 如图\ref{fig:processor}. \par
    \begin{figure}[htbp]
        \centering
        \includegraphics[width=\textwidth]{pictures/processor.png}
        \caption{processor}
        \label{fig:processor}
    \end{figure}
    首先, 集群地面站以UDP的方式向每一架无人机广播飞行指令(内部有指定接收的飞机编号), 由通信模块(communication)下的UDP$\_$Receive子功能进行接收, 同时将其发送到主控(processor). 
    processor 实例化之后, 会从实时子系统中获取当前无人机的编号id, 并据此id从对个QGC$\_$COMMAND中筛选出属于自己的指令. 
    获取任务指令之后, 进行具体的解析, 获取飞行的任务指令, 同时判断数据的有效性, 且为UDP$\_$Send发送一个ACK标志位, 来对集群地面站产生一个应答响应. 
    获取有效的飞行控制指令(single, formation, or self-organizing), 且指定与硬件的交互方式(OFFBOARD or ALTCTL(RC)), 
    将指令整合发布给计算逻辑单元下的各个控制器, 进行计算, 最后将计算结果从串口发送给硬件PX4. 
    
\subsection{通信模块}
    通信模块在整个集群之间担任着非常重要的地位, 有了它我们就可以完成多个个体之间的交流配合. \par
如图\ref{fig:comm}所示内部主要功能有 UDP$\_$Send 和 UDP$\_$Receive 两大功能, 其中 UDP$\_$Send 主要担任着广播自己的位姿信息, 供机间之间配合调用. 有了send, receive 也是必不可少的. UDP$\_$Receive 主要是用来接收集群地面站的指令信息以及其余飞机广播的位姿信息, 进而数据转发给processor进行处理. 
\begin{figure}[htbp]
    \centering
    \includegraphics[width=\textwidth]{pictures/communication.png}
    \caption{communication}
    \label{fig:comm}
\end{figure}
\subsection{导航控制器}
    \chapter{navigator}

\section{Waypoints List}

\subsection{enu Waypoints List}
该文本存放关于enu航点列表的描述
\clearpage
\subsection{wgs84 Waypoints list}
\clearpage

\section{Waypoints Changed Logic}

\subsection{normals to normals}
\clearpage

\subsection{normals to loiters}
\clearpage
\subsection{loiters to loiters}
\clearpage

\subsection{loiters to normals}
\clearpage

\section{Publishing date to position controller}
\clearpage


\subsection{制导控制器}
    制导是导引和控制飞行器按一定规律飞向目标或预定轨道的技术和方法。制导过程中,导引系统不断测定飞行器与目标或预定轨道的相对位置关系,发出制导信息传递给飞行器控制系统,以控制飞行。分有线制导、无线电制导、雷达制导、红外制导、激光制导、音响制导、地磁制导、惯性制导和天文制导等。
        如果说导航是给出航线,引导目标到达某点,那么制导就是导引并控制目标到达某点。区别就像别人问路,导航就是给他指路,制导就是给他带路
        导航的对象范围比较大,通常用他的狭义定义指的是载具设备,但是广义定义下也能对人;制导的对象则范围小,通常用的狭义定义专指武器(导弹,制导炸弹,鱼雷),广义下也加入了飞行器
\subsection{姿态控制器}
    \chapter{attitude controller}

\section{offboard}


\subsection{mavros发送offboard数据流}
\begin{lstlisting}[title=publish messgae, frame=shadowbox]
    // 1. offboard publish (mavros topic)
    fixed_wing_local_att_sp_pub = nh.advertise<mavros_msgs::AttitudeTarget>("mavros/setpoint_raw/attitude", 10);
    
    #define MAVLINK_MSG_ID_SET_ATTITUDE_TARGET 82
      typedef struct __mavlink_set_attitude_target_t {
          uint32_t time_boot_ms; /*< [ms] Timestamp (time since system boot).*/
          float q[4]; /*<  Attitude quaternion (w, x, y, z order, zero-rotation is 1, 0, 0, 0)*/
          float body_roll_rate; /*< [rad/s] Body roll rate*/
          float body_pitch_rate; /*< [rad/s] Body pitch rate*/
          float body_yaw_rate; /*< [rad/s] Body yaw rate*/
          float thrust; /*<  Collective thrust, normalized to 0 .. 1 (-1 .. 1 for vehicles capable of reverse trust)*/
          uint8_t target_system; /*<  System ID*/
          uint8_t target_component; /*<  Component ID*/
          uint8_t type_mask; /*<  Mappings: If any of these bits are set, the corresponding input should be ignored: bit 1: body roll rate, bit 2: body pitch rate, bit 3: body yaw rate. bit 4-bit 6: reserved, bit 7: throttle, bit 8: attitude*/
     }) mavlink_set_attitude_target_t;
\end{lstlisting}
\section{Coordinated Turn 协调转弯}
\subsection{not being wind}
方向角的变化率是和机体的roll以及倾斜角(bank angle)有关系, 我们需要寻找一个简单的关系来帮助我们研究这种线性传递函数的关系 -- 协调转弯. \par
在协调转弯期间, 飞机在体坐标系下没有横向加速度. 从分析的角度来看, 协调转弯的一个假设运行我们得到一个简单的表达式将 heading rate 和 bank angle 联系起来. \par
协调转弯时, 为了无人机没有侧向力, bank angle $\phi$ 被设置.
在图\ref{fig:1}中, 作用在微型飞行器上的离心力与作用在水平方向上的升力的水平分量相等并相反。
\begin{figure}[htpb]
    \centering
    \includegraphics[width=0.8\textwidth]{pictures/5_1.png}
    \caption{爬升协调转弯MAV上的力}
    \label{fig:1}
\end{figure}
\par 作用在水平方向力的关系表示如下: 
\begin{align}
    F_{lift} sin \phi cos (\chi - \psi) &= m \frac{v^{2}}{R} \nonumber \\
    &= m v \omega \nonumber \\
    &= m (V_{g} cos \gamma) \dot{\chi} 
    \label{equ:1}
\end{align}
其中, $F_{lift}$代表的是升力, $\gamma$ 代表的是飞行轨迹的角度, $V_{g}, \chi$ 分别表示的是地速度以及方向角. \textcolor{red}{向心加速度的表达式: $a_{n} = \frac{v^{2}}{R} = v \omega$}
\par 离心力(The centrifugal force)(\textcolor{red}{$m (V_{g} cos \gamma) \dot{\chi} $})计算的时候, 用到了在惯性坐标系$k^{i}$上的方向角变化率$\dot{\chi}$ 和 速度的水平分量 $V_{a}cos \gamma$
\par 同样, 升力的垂直分量与重力在 $j^{b} - k^{b}$平面上的投影是等大反向的. 
垂直方向上的合力为:
\begin{equation}
    F_{lift} cos \phi = mg cos\gamma
    \label{equ:2}
\end{equation}
将等式\ref{equ:1}除以\ref{equ:2}得的 $\dot{\chi}$
\begin{equation}
    \dot{\chi} = \frac{g}{V_{g}} tan \phi cos(\chi - \psi)
    \label{equ:3}
\end{equation}
等式\ref{equ:3}就是协调转弯的表达式. 
\par 考虑到转弯半径等于 \textcolor{blue}{ $R = V_{g} \frac{cos \gamma}{\dot{\chi}}$}, 将上式代入半径中, 得到式子\ref{equ:4}. 在没有风或侧滑的情况下, 有\textcolor{red}{$V_{a} = V_{g}$和$\psi = \chi$}, 从而得到了式子\ref{equ:5}. 
\begin{equation}
    R = \frac{V_{g}^{2} cos \gamma}{g tan \phi cos(\chi - \psi)} 
    \label{equ:4}
\end{equation}
\begin{equation}
    \dot{\chi} = \frac{g}{V_{g}} tan \phi = \dot{\psi} = \frac{g}{V_{a}} tan \phi
    \label{equ:5}
\end{equation}
\par 在 9.2 节中, 我们将要介绍 在有风的情况下 \textcolor{blue}{$ \dot{\psi} = \frac{g}{V_{a}} tan \phi$} 该式子也成立
\clearpage
\subsection{being wind-Kinematic Model of Controlled Flight}
% 控制飞行动力学模型\par
在推导降阶表达式中, 简化的目的是估计运动中力平衡以及动量平衡的关系式(这些包含了 $\dot{u}, \dot{v}, \dot{\omega}, \dot{p}, , \dot{q}, \dot{r}$), 预估这些变量需要计算复杂的空气动力. 这些变量表达式可以被更简单的动力学表达式替代. 
这个更简单的动力学表达式是\textcolor{blue}{针对协调转弯和加速爬升的特定飞行条件而导出}.
\begin{figure}[htpb]
    \centering
    \includegraphics[width=0.8\textwidth]{pictures/2_10.png}
    \caption{航线轨迹角度$\gamma$和航向角$\chi$}
    \label{fig:2_10}
\end{figure}
针对图\ref{fig:2_10}, 飞机相对于惯性系的速度矢量可以用航向角和(惯性参考)飞行路径角表示为 
\begin{gather} % 输入多行公式
    V_{g}^{i} = V_{g} \begin{pmatrix}
        cos \chi cos \gamma \\
        sin \chi cos \gamma \\
        -sin \gamma \\
      \end{pmatrix}
      = \begin{pmatrix}
        \dot{p_{n}} \\
        \dot{p_{e}} \\
        \dot{h} \\
      \end{pmatrix}
      \label{equ:6}
  \end{gather}
\par 由于控制飞机的航向和空速是很常见的,因此用$\psi$和$V_{a}$表示等式\ref{equ:6}很有用. 
\begin{gather} % 输入多行公式
    V_{g} \begin{pmatrix}
        cos \chi cos \gamma \\
        sin \chi cos \gamma \\
        -sin \gamma \\
      \end{pmatrix} - \begin{pmatrix}
        w_{n} \\
        w_{e} \\
        w_{d} \\
      \end{pmatrix} =  V_{a} \begin{pmatrix}
        cos \psi cos \gamma_{a} \\
        sin \psi cos \gamma_{a} \\
        -sin \gamma_{a} \\
      \end{pmatrix}
      \label{equ:wind}
  \end{gather}
  结合风的表达式\ref{equ:wind}(地速等于空速加风速, 
    其中的 $\gamma_{a}$ 代表的是 空速的方向和水平方向的夹角), 我们可以得到
  \begin{gather} % 输入多行公式
    \begin{pmatrix}
        \dot{p_{n}} \\
        \dot{p_{e}} \\
        \dot{h} \\
      \end{pmatrix} = V_{a} \begin{pmatrix}
        cos \psi cos \gamma_{a} \\
        sin \psi cos \gamma_{a} \\
        sin \gamma_{a} \\
      \end{pmatrix} +  \begin{pmatrix}
        w_{n} \\
        w_{e} \\
        -w_{d} \\
      \end{pmatrix}
      \label{equ:7}
  \end{gather}
  如果我们假设飞机保持在一个恒定的高度,并且没有向下的风分量,那么运动学表达式简化为\ref{equ:8}, 同样该模型也是无人机领域中比较常用的模型. 
  \begin{gather} % 输入多行公式
    \begin{pmatrix}
        \dot{p_{n}} \\
        \dot{p_{e}} \\
        \dot{h} \\
      \end{pmatrix} = V_{a} \begin{pmatrix}
        cos \psi \\
        sin \psi \\
        0 \\
      \end{pmatrix} +  \begin{pmatrix}
        w_{n} \\
        w_{e} \\
        0 \\
      \end{pmatrix}
      \label{equ:8}
  \end{gather}
  \subsection{Coordinated Turn}
  之前的协调转弯的表达式为 $\dot{\chi} = \frac{g}{V_{g}} tan \phi cos(\chi - \psi)$. 
  即使在第6章中描述的自动驾驶回路并没有强制执行协调转弯条件,
  飞机必须倾斜才能转弯(而不是打滑才能转弯)这个基本条件已经被这个模型捕捉到了。\par
  协调转弯可以被 heading 和 空速进行表示. 我们先对\ref{equ:wind}两边进行求导, 得到下面的式子\ref{equ:9}
  \begin{gather} % 输入多行公式
    \begin{pmatrix}
        cos \chi cos \gamma & - V_{g} sin \chi cos \gamma & - V_{g} cos \chi sin \gamma \\
        sin \chi cos \gamma & V_{g} cos \chi cos \gamma & - V_{g} sin \chi sin \gamma \\
        -sin \gamma & 0 & -cos \gamma \\
      \end{pmatrix} \begin{pmatrix}
        \dot{V_{g}} \\
        \dot{\chi} \\
        \dot{\gamma} \\
    \end{pmatrix}
      = \begin{pmatrix}
        cos \psi cos \gamma_{a} & - V_{a} sin \psi cos \gamma_{a} & - V_{a} cos \psi sin \gamma_{a} \\
        sin \psi cos \gamma_{a} & V_{a} cos \psi cos \gamma_{a} & - V_{a} sin \psi sin \gamma_{a} \\
        -sin \gamma_{a} & 0 & -cos \gamma_{a} \\
      \end{pmatrix} \begin{pmatrix}
        \dot{V_{a}} \\
        \dot{\psi} \\
        \dot{\gamma_{a}} \\
    \end{pmatrix}
      \label{equ:9}
  \end{gather}
  \par 在定高和没有向下风分量的情况下, $\gamma, \gamma_{a}, \dot{\gamma}, \dot{\gamma_a}$ 和 $w_{d}$ 都是0, 根据$\dot{V_{a}}$ 和$\dot{\chi}$求解$\dot{V_{g}}$ 和$\dot{\psi}$
  \begin{equation}
    \begin{split}
      \dot{V_{g}} &= \frac{\dot{V_{a}}}{cos (\chi - \psi)} + V_{g} \dot{\chi} tan(\chi - \psi) \\
      \dot{\psi} &= \frac{\dot{V_{a}}}{V_{a}} tan (\chi - \psi) + \frac{V_{g} \dot{\chi}}{V_{a}cos(\chi - \psi)}
    \end{split}
\end{equation}
\par 若假定空速为常数, 那么得\ref{equ:10} 最值得注意的是在有风的情况下,这个等式是成立的。
\begin{equation}
    \dot{\chi} = \frac{g}{V_{g}} tan \phi 
    \label{equ:10}
\end{equation}
\subsection{px4内部的实现}
第一次处理产生\ref{equ:att:turn:1}, 得到$roll_{constrained}$, 之后在对其进行$(-roll_{setpoint}, roll_{setpoint})$约束. 得出\ref{equ:att:turn:2}, 进而进行 PID 控制, 产生\ref{equ:att:turn:3}.
\begin{equation}
  roll_{constrained}=
  \begin{cases}
  constrained[-80^{o}, 80^{o}], &fabs(roll_{current} < 90^{o}) \\
  constrained[100^{o}, 180^{o}], &fabs(roll_{current} > 90^{o}) \& roll_{current} > 0^{o}\\
  constrained[-180^{o}, -100^{o}], &fabs(roll_{current} > 90^{o}) \&roll_{current} < 0^{o}
  \end{cases}
  \label{equ:att:turn:1}
\end{equation} \\
\begin{equation}
  roll_{constrained} =roll_{constrained}.constrained[-roll_{setpoint}, roll_{setpoint}]
  \label{equ:att:turn:2}
\end{equation}\\
\begin{equation}
  \begin{split}
    \dot{yaw} &= \frac{tan(roll_{constrained}) * cos(pitch_{current}) * G}{V_{air}} , (V_{air} = V_{air} < V_{air}^{min} ? V_{air}^{min} : V_{air}) \\
    \dot{roll} &= \frac{roll_{setpoint} - roll_{current}}{0.1} \\
    \dot{pitch} &= \frac{pitch_{setpoint} - pitch_{current}}{0.1}
    \label{equ:att:turn:3}
  \end{split}
\end{equation}
在第一次计算的基础上, 续进行第二手我们继计算, 在px4内部, 首先实现进行参数的设置(见下面的"参数设置"), 
\begin{equation}
  \begin{split}
  fw\_acro\_x\_max &= 90^{o} \\
  fw\_acro\_y\_max &= 90^{o} \\
  fw\_acro\_z\_max &= 45^{o}
  \end{split}
  \label{equ:config}
\end{equation}
\begin{lstlisting}[title=参数设置, frame=shadowbox]
  _roll_ctrl.set_max_rate(radians(_param_fw_acro_x_max.get()));
  _pitch_ctrl.set_max_rate_pos(radians(_param_fw_acro_y_max.get()));
  _pitch_ctrl.set_max_rate_neg(radians(_param_fw_acro_y_max.get()));
  _yaw_ctrl.set_max_rate(radians(_param_fw_acro_z_max.get()));
\end{lstlisting}
\begin{figure}[htbp]
  \centering
  \subfigure[x]{
      \includegraphics[width=0.48\textwidth]{pictures/parameter1.png}
  }
  \subfigure[y and z]{
      \includegraphics[width=0.48\textwidth]{pictures/att_parameter2.png}
  }
  \caption{parameters}
  \label{fig:param}
\end{figure}
其中各个参数的描述见\ref{fig:param}, 在px4中分别对应的值为\ref{equ:config}. 实现了各个参数的初始化之后, 进行下面的处理
\begin{equation}
  \begin{split}
  roll_{bodyrateSetpoint} &= [\dot{roll} - sin(pitch_{current}) * \dot{yaw}]\\
                          &.constrained[-FW\_ACRO\_X\_MAX, FW\_ACRO\_X\_MAX] \\    
  pitch_{bodyrateSetpoint} &= [cos(roll_{current})*\dot{roll} + cos(pitch_{current}) * sin(roll_{current}) * \dot{yaw}]\\
                          &.constrained[-FW\_ACRO\_Y\_MAX, FW\_ACRO\_Y\_MAX]\\
  yaw_{bodyrateSetpoint} &= [-sin(roll_{current}) * \dot{pitch} + cos(roll_{current}) * cos(pitch_{current}) * \dot{yaw}] \\
                          &.constrained[-FW\_ACRO\_Z\_MAX, FW\_ACRO\_Z\_MAX]
  \end{split}
\end{equation}
最终将上面约束过的体变化率当做姿态的设定值, 发布给下面的控制器以及执行器
\begin{equation}
  \begin{split}
    roll_{setpoint} &= roll_{bodyrateSetpoint} \\
    pitch_{setpoint} &= pitch_{bodyrateSetpoint} \\
    yaw_{setpoint} &= yaw_{bodyrateSetpoint}  
  \end{split}
\end{equation}
\subsection{欧拉角, 四元数的相互转换}
\subsubsection{欧拉角转换为四元数}

\begin{lstlisting}[title=欧拉角转换为四元数]
  void euler_2_quaternion(float angle[3], float quat[4])
  {
      // q0 q1 q2 q3
      // w x y z
      double cosPhi_2 = cos(double(angle[0]) / 2.0);
      double sinPhi_2 = sin(double(angle[0]) / 2.0);
      double cosTheta_2 = cos(double(angle[1]) / 2.0);
      double sinTheta_2 = sin(double(angle[1]) / 2.0);
      double cosPsi_2 = cos(double(angle[2]) / 2.0);
      double sinPsi_2 = sin(double(angle[2]) / 2.0);
      
      quat[0] = float(cosPhi_2 * cosTheta_2 * cosPsi_2 + sinPhi_2 * sinTheta_2 * sinPsi_2);
      quat[1] = float(sinPhi_2 * cosTheta_2 * cosPsi_2 - cosPhi_2 * sinTheta_2 * sinPsi_2);
      quat[2] = float(cosPhi_2 * sinTheta_2 * cosPsi_2 + sinPhi_2 * cosTheta_2 * sinPsi_2);
      quat[3] = float(cosPhi_2 * cosTheta_2 * sinPsi_2 - sinPhi_2 * sinTheta_2 * cosPsi_2);
  }  
\end{lstlisting}

\subsubsection{四元数转换为欧拉角}
\begin{lstlisting}[title=四元数转换为欧拉角]
  Quaternion(const Euler<Type> &euler)
    {
        Quaternion &q = *this;
    
        Type cosPhi_2 = Type(cos(euler.phi() / Type(2)));
        Type cosTheta_2 = Type(cos(euler.theta() / Type(2)));
        Type cosPsi_2 = Type(cos(euler.psi() / Type(2)));
        Type sinPhi_2 = Type(sin(euler.phi() / Type(2)));
        Type sinTheta_2 = Type(sin(euler.theta() / Type(2)));
        Type sinPsi_2 = Type(sin(euler.psi() / Type(2)));
        q(0) = cosPhi_2 * cosTheta_2 * cosPsi_2 +
               sinPhi_2 * sinTheta_2 * sinPsi_2;
        q(1) = sinPhi_2 * cosTheta_2 * cosPsi_2 -
               cosPhi_2 * sinTheta_2 * sinPsi_2;
        q(2) = cosPhi_2 * sinTheta_2 * cosPsi_2 +
               sinPhi_2 * cosTheta_2 * sinPsi_2;
        q(3) = cosPhi_2 * cosTheta_2 * sinPsi_2 -
               sinPhi_2 * sinTheta_2 * cosPsi_2;
}
\end{lstlisting}

\subsection{针对offboard对px4的更改}
在 \textcolor{red}{$fw\_att\_control.cpp$} 控制器中, 对yaw重新计算的时候, 使用的是 $roll_{setpoint}$ 而不是根据当前的roll进行两次约束之后得到的结果. 同时
也去掉$cos(pitch_{current})$这一项.

\section{RC control}
\subsection{raspberry's handled logic and px4's receiver}
树莓派的控制指令以mavlink消息$(MAVLINK_MSG_ID_RC_CHANNELS_OVERRIDE = 70)$进入到px4内部, 处理函数名字为:
\begin{lstlisting}[title=mavlink处理函数声明体]
    // Firmware/src/modules/mavlink/mavlink_receiver.cpp
    void MavlinkReceiver::handle_message(mavlink_message_t *msg);
\end{lstlisting}
\par 处理函数内部会对信息进行解码处理,得到18个通道(channels)的值,并且赋值给 $input\_rc\_s$ rc{}变量, 之后进行
有效性的处理(判断值18个通道的值是否为65535或者是0, 若是, 则该通道的值变为0; 反之, 该通道的值不变,且更新$rc.channel\_count$的值),
处理之后, 使用发布器 $\_rc\_pub$ 将该变量发布到 \\ $ORB\_ID(input\_rc)$话题上(定义发布器的时候, 会设置到PublicationMulti机制, 实例化会有一个优先级的赋值, 可以参见\textcolor{red}{需要补充} )
树莓派发送的四个值占用前四个通道,这个四个通道对应的顺序和遥控器对应指令的通道是不一样的 $(roll = 1 , pitch = 2, throttle = 3, yaw = 4)$.
\begin{lstlisting}[title=$input\_rc\_s$结构体的定义]
    struct input_rc_s {
        uint64_t timestamp;
        uint64_t timestamp_last_signal;
        uint32_t channel_count;
        int32_t rssi;
        uint16_t rc_lost_frame_count;
        uint16_t rc_total_frame_count;
        uint16_t rc_ppm_frame_length;
        uint16_t values[18];
        bool rc_failsafe;
        bool rc_lost;
        uint8_t input_source;
        uint8_t _padding0[3]; // required for logger
    }
    // publisher declaration 
    uORB::PublicationMulti<input_rc_s>			_rc_pub{ORB_ID(input_rc), ORB_PRIO_LOW};
\end{lstlisting}
\begin{lstlisting}[title=树莓派以及固件通道定义]
    // raspberry
    #define PITCH_CHANNEL 		1
    #define ROLL_CHANNEL 		2
    #define YAW_CHANNEL 		3
    #define THROTTLE_CHANNEL 	4
    
    // px4
    #define PITCH_CHANNEL 		1
    #define ROLL_CHANNEL 		2
    #define YAW_CHANNEL 		3
    #define THROTTLE_CHANNEL 	4
\end{lstlisting}

\subsection{其他部分的订阅}

mavlink接收函数拿到数据之后,进行一些处理之后, 将数据publish出去。在$rc\_update$中进行订阅
(multi publish 及其 subscribe 机制在后续文章中进行讲解),订阅器定义如下:
\begin{lstlisting}[title=订阅器的声明定义]
    // src/modules/sensors/rc_update.h
    uORB::Subscription	_rc_sub{ORB_ID(input_rc)};				/**< raw rc channels data subscription */
\end{lstlisting}

在源文件中$rc\_poll\ api$中进行更新获取该变量的值,再进行一些逻辑标志位的检索,
最后对其进行第二次的有效性检测(将每个通道的值约束到某一个固定的区间, 类似constrained()函数; trim操作, 
准备数据为manual id publish做准备), 再一次更新元素的值. 更新后的值处理.
\begin{itemize}
    \item [(1)] 发布到 ORB\_ID(rc\_channels) 话题上面.
    \item [(2)] 将 struct manual\_control\_setpoint\_s manual = {} 数据进行约束到[-1.0f, 1.0f]之间更新到该变量中,继而 发布到 ORB\_ID(manual\_control\_setpoint) 消息上.
    \item [(3)] 将第二次发布得到的 manual 变量, 赋值给 actuator\_group\_3.control 数组, 发布到\\ ORB\_ID(actuator\_controls\_3)消息上.
\end{itemize}

我们真正关心的消息topic应该是$(1\>, 2\>)$, 也就是 $ORB\_ID(rc\_channels)$ 和 $ORB\_ID(manual\_control\_setpoint)$, 
其中 $ORB\_ID(rc\_channels)$ 会被作用到PWM(遥控器和接收机的通信方式)上,所以,我们只需要关心的就是\\ $ORB\_ID(manual\_control\_setpoint)$ 消息topic即可.
\par
上述变量及其函数定义如下:
\begin{lstlisting}[title=上述变量及其函数定义]
    void RCUpdate::rc_poll(const ParameterHandles &parameter_handles);


    rc_channels_s _rc {};			/**< r/c channel data */
    orb_publish_auto(ORB_ID(rc_channels), &_rc_pub, &_rc, &instance, ORB_PRIO_DEFAULT);
    
    
    struct manual_control_setpoint_s manual = {};
    orb_publish_auto(ORB_ID(manual_control_setpoint), &_manual_control_pub, &manual, &instance,
                ORB_PRIO_HIGH);
    
    
    /* copy from mapped manual control to control group 3 */
    struct actuator_controls_s actuator_group_3 = {};
    /* publish actuator_controls_3 topic */
    orb_publish_auto(ORB_ID(actuator_controls_3), &_actuator_group_3_pub, &actuator_group_3, &instance,
                 ORB_PRIO_DEFAULT);
\end{lstlisting}

\subsection{ORB\_ID(manual\_control\_setpoint)}
这个uORB消息, 在px4内部会被 FixedWing Position Controller , FixedWing Attitude Controller 及其他原件进行订阅使用, 
这里我们需要关心的 FixedWing Position Controller , FixedWing Attitude Controller中的使用情况.
\subsubsection{FixedWing Position Controller}
在制导控制器中, px4会根据当前的throttle期望值, 调用 内部的 TECS, 进行新的姿态设定值, 计算期望空速, pitch, 以及使用其他的逻辑来进行计算 yaw, 以及roll的设定值, 赋值给变量$\_att\_sp$, 从而在最后发布给下一层的姿态控制器. 
\begin{lstlisting}[title=计算一些姿态的设定值]
_att_sp.roll_body = _manual.y * _parameters.man_roll_max_rad;
_att_sp.yaw_body = 0;

const float deadBand = 0.06f;
float factor = 1.0f - deadBand;
float pitch = -(_manual.x + deadBand) / factor;

// calculate the demanded airspeed.
float
FixedwingPositionControl::get_demanded_airspeed()
{
	float altctrl_airspeed = 0; // the demanded airspeed.

	// neutral throttle corresponds to trim airspeed
	if (_manual.z < 0.5f) {
		// lower half of throttle is min to trim airspeed
		altctrl_airspeed = _parameters.airspeed_min +
				   (_parameters.airspeed_trim - _parameters.airspeed_min) *
				   _manual.z * 2;

	} else {
		// upper half of throttle is trim to max airspeed
		altctrl_airspeed = _parameters.airspeed_trim +
				   (_parameters.airspeed_max - _parameters.airspeed_trim) *
				   (_manual.z * 2 - 1);
	}

	return altctrl_airspeed;
}
\end{lstlisting}

\textcolor{red}{上述代码公式转换如下}

\subsubsection{FixedWing Attitude Controller}
姿态控制器拿到数据且赋值给 $\_manual$ 变量.
\begin{lstlisting}[title=高度处理逻辑]
if (_vcontrol_mode.flag_control_rattitude_enabled) {
	if (fabsf(_manual.y) > _parameters.rattitude_thres || fabsf(_manual.x) > _parameters.rattitude_thres) {
		_vcontrol_mode.flag_control_attitude_enabled = false;
	}
}
\end{lstlisting}
\par
\begin{itemize}
    \item 若该变量的y和x大于 $\_parameters.rattitude\_thres$ 参数的值, 则 $flag\_control\_attitude\_enabled = false$, 若这个时候$flag\_control\_rates\_enabled$ 为真, 那么执行处理逻辑1; 再将值发布到\\$ORB\_ID(vehicle\_rates\_setpoint)$消息上, 进行下一步的处理.
    \item 若该变量的y和x小于等于 $\_parameters.rattitude\_thres$ 参数的值, 则 $flag\_control\_attitude\_enabled = true$, 执行处理逻辑2, 将值publish到 $\_attitude\_setpoint\_id$ 上面(这个topic就对应offboard从mavros发送到px4的控制逻辑层)
    \item 若不符合上面两个逻辑, 直接执行处理逻辑3, 将控制指令直接发布给执行器. 将值publish到 \\ $\_attitude\_setpoint\_id$ 上
\end{itemize}
\begin{lstlisting}[title=处理逻辑1]  
    _rates_sp.roll = _manual.y * _parameters.acro_max_x_rate_rad;
    _rates_sp.pitch = -_manual.x * _parameters.acro_max_y_rate_rad;
    _rates_sp.yaw = _manual.r * _parameters.acro_max_z_rate_rad;
    _rates_sp.thrust_body[0] = _manual.z;        
\end{lstlisting}
\begin{lstlisting}[title=处理逻辑2]  
    // STABILIZED mode generate the attitude setpoint from manual user inputs
					_att_sp.timestamp = hrt_absolute_time();

					// calculate the setpoints 
					_att_sp.roll_body = _manual.y * _parameters.man_roll_max + _parameters.rollsp_offset_rad;
					_att_sp.roll_body = math::constrain(_att_sp.roll_body, -_parameters.man_roll_max, _parameters.man_roll_max);
					_att_sp.pitch_body = -_manual.x * _parameters.man_pitch_max + _parameters.pitchsp_offset_rad;
					_att_sp.pitch_body = math::constrain(_att_sp.pitch_body, -_parameters.man_pitch_max, _parameters.man_pitch_max);
					_att_sp.yaw_body = 0.0f;
					_att_sp.thrust_body[0] = _manual.z;

					// get the Quatf
					Quatf q(Eulerf(_att_sp.roll_body, _att_sp.pitch_body, _att_sp.yaw_body));
					q.copyTo(_att_sp.q_d);
                    _att_sp.q_d_valid = true;  
                    
                    _attitude_setpoint_id = ORB_ID(vehicle_attitude_setpoint);
\end{lstlisting}
\begin{lstlisting}[title=处理逻辑3]  	
    /* manual/direct control */
    _actuators.control[actuator_controls_s::INDEX_ROLL] = _manual.y * _parameters.man_roll_scale + _parameters.trim_roll;
    _actuators.control[actuator_controls_s::INDEX_PITCH] = -_manual.x * _parameters.man_pitch_scale + _parameters.trim_pitch;
    _actuators.control[actuator_controls_s::INDEX_YAW] = _manual.r * _parameters.man_yaw_scale + _parameters.trim_yaw;
    _actuators.control[actuator_controls_s::INDEX_THROTTLE] = _manual.z;

    _actuators_id = ORB_ID(actuator_controls_0);
\end{lstlisting}

\begin{lstlisting}[title=$\_attitude\_setpoint\_id$]
_attitude_setpoint_id = ORB_ID(vehicle_attitude_setpoint);
\end{lstlisting}

\subsection{Low pass filter}
\begin{lstlisting}[title=Low pass filter from px4]
    float LowPassFilter2p::apply(float sample)
{
	// do the filtering
	float delay_element_0 = sample - _delay_element_1 * _a1 - _delay_element_2 * _a2;

	if (!PX4_ISFINITE(delay_element_0)) {
		// don't allow bad values to propagate via the filter
		delay_element_0 = sample;
	}

	const float output = delay_element_0 * _b0 + _delay_element_1 * _b1 + _delay_element_2 * _b2;

	_delay_element_2 = _delay_element_1;
	_delay_element_1 = delay_element_0;

	// return the value. Should be no need to check limits
	return output;
}
\end{lstlisting}
    
        \clearpage
    \section{编码和单元测试}
      本项目代码采用面向对象高级编程语言C++编写而成. 采用CMake编译构建系统. 