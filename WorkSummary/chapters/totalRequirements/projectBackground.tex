无人飞行器(Unmanned Air Vehicle.UAV)具有广阔的应用前景,是近年来高技术研究的热点目标之一。
    随着社会的发展和经济的进步,无人机的发展逐步趋
于成熟。无人机的应用有不同的场景和方式,例如,应用于
军事勘探和追踪,农药的无人喷洒,民用影像航拍等。以
2017 年发生飓风的波多黎各为例,利用最合理的方法向
该地投放物资. 同时也伴随着计算机技术,通信技术,传感器技术,电池技术等的
    飞速发展,开展微型UAV研究并把它运用到军事或民用中已经成为可能。
    本论文以PX4控制逻辑为基础,进行了一定的改进,实现了更加平滑的转弯控制,使得
    无人机能以更地高效率执行航线。\par
    无人机系统通常较有人机复杂。系统的基本组成主要包括无人机、地面控制站、
发射回收装置及地面数据终端。控制站通常提供三个工作站:完成任务控制、无人
机控制及图像分析。地面数据终端完成与无人机的信息交流。另外,有些无人机系
统还增加了卫星作为其系统的组成部分,增加GPS以提高无人机定位的精度。无人
机研究的关键技术包括:飞行器系统技术一用于预测故障的综合性管理系统和满足
故障容错要求的制导、导航和控制系统;人工智能技术一用于解决无人机的自主程
度问题;通信技术一宽带、大数据量的数据链技术可以使无人机远距离快速传递信
息,实施超视距控制.