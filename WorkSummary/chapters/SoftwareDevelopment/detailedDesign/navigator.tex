    导航就字面上说,就是引导航行的意思,而其确切的定义可表述为:导航是有目的地、安全有效地引导运动体(船只、潜艇、地面车辆以及飞机、宇宙飞船等)
        从一地到另一地的控制过程。
        所有导航系统的相关研究,都是为了解决三个基本的导航课题。这三个课题是:
        \begin{itemize}
            \item [(1)] 
                如何确定被导航对象的位置
            \item [(2)]
                如何确定被导航对象从一个位置到另一个位置前进的方向
            \item [(3)]
                如何确定距离(或者速度、时间)
        \end{itemize}
        \par对每个导航系统来说,就是利用导航手段不断确定被导航对象航行中的位置、方向、距离、时间和速度,这些通常称之为“导航参量”。
        在这些导航参量中,对慢速运动体来说或对于远距离航行来说,“位置”是关键。因为导航系统知道了“在哪儿”之后,就可以决定是继续保持当时的速度和航向,还
    是要作某种改变。因此,按传统的观点,导航系统从某种意义上说就是定位系统。
但是对于高速运动的导航对象来说,测量和判断之间的时间滞后,使得位置信息不
具有更多的意义。这种情况下,驾驶员最关心的导航参量就是“航向”和“距离”,以
决定“到终点或下一条航路点要经过的那条航线?还有多远”的问题:而如果交通密
度很高,还会产生这样的问题:“在这个时刻,我应该在哪里?我实际上在哪里?怎
样到达我应该到达的下一个位置上去?”因此,这时所需要的连续的、实时的驾驶信
息输出,以便通过制导计算机来实行自动操纵。
总之,由于导航的目的和对象的不同,要求解决的问题也会有所区别。但从根
本上说,导航就是为了提供航行中的位置、方向、距离和速度这些导航参量。因此,
导航的研究,就是要弄清楚这些导航参量的如何测量和如何运用;而导航的实践,
就是运用所得到的结果来保证运动体安全而有效地航行。
从所用技术来分,导航可分为惯性导航、雷达、卫星导航;导航台;从应用方式来分,可分为自丰导航和地面导航。上面的惯性导航和雷达为自主导航;导航台
为地面导航;而卫星导航是一种综合的导航手段。
\par
由此可知,如何确定位置是导航需要解决的基本问题之一,而且传统的导航系统从某种意义上说就是一个定位系统,这说明了定位技术在导航过程中的重
要地位。
目前,对于地面移动目标的定位技术主要包括无线电定位技术、GPS导航定位技术、惯性导航定位技术。
\begin{itemize}
    \item [1)] 无线电定位技术 \par
        该种定位技术由超高频视距无线电系统构成的同步时序网络,采用跳频和时分
    多址技术,提供数百个地面和空中平台的实时位置轨迹。定位精度可达几十米内,
    缺点是建网成本太高,军事上应用居多.
    \item [2)] 惯性导航定位技术 \par
    惯性导航系统利用陀螺、加速度计等惯性元件,测量运动平台相对于惯性空间的线运动和角运动参数,在给定初始条件下,由计算机推算出平台的导航参数,以
    引导平台完成预定的航行或行驶任务。惯性导航的突出优点是,具有高度的自主性,不需要外界的帮助,就能独立完成导航任务。缺点是有自身无法避免的累积误差.
    \item [3)] 基于GPS系统的定位技术 \par
    该技术优点是定位精度高,定位半径可达十几米。目前没有一种传统的导航定位技术能够达到GPS这样的高精度、高速度、全天候和全球性的性能。
    其缺点是需要终端内置GPS接收机,定位精度受终端所在环境的影响较大,如用户在室内时,定位精度将降低,甚至无法完成定位。
\end{itemize}
\par 从以上介绍可知这些导航定位技术各有其优点和不足,无线电定位技术和GPS
导航定位技术可以获得高精度,但容易受到干扰;惯性导航定位装置是纯自主式的,但必须借助于里程计等定时进行校正,才能保证定位精度。
因此,目前地面平台往往采用两种或两种以上的导航装置组成综合导航定位系统。但是组合导航系统是以提高成本为代价的,目前国内,应用较多也比较实用的是结合GIS的地图匹配算法.
\par
导航需要解决的第二个问题是在当前位置已知的前提下,如何才能到达目的地,也即路径规划。
