\documentclass[UTF8,a4paper,10pt,nocolorlinks]{ctexart}
\usepackage[left=2.50cm, right=2.50cm, top=2.50cm, bottom=2.50cm]{geometry} %页边距
\CTEXsetup[format={\Large\bfseries}]{section} %设置章标题居左   
\usepackage{ctex}
\CTEXoptions[today=old]
\usepackage{cite}
% 代码块儿
\usepackage{textcomp} % 必须加上,否则报错
\usepackage{listings}
\usepackage{xcolor}
% \usepackage{fontspec}
% \setmonofont{Consolas}
\usepackage{varioref}       % ref 跨页调用
\usepackage{ctex}
\usepackage{multicol}
\usepackage{amssymb}        % 等于号 上面 加一个三角形
\usepackage{setspace}
\usepackage{tikz} % package used for the tikz
\usepackage{mdframed}
\usepackage{titletoc}
\usepackage{etoolbox}

\usepackage{helvet}
\usepackage{caption}
\usepackage{multicol} %用于实现在同一页中实现不同的分栏
\usepackage{changepage}
\usepackage{graphics}
\usepackage{amsmath, amsfonts, amssymb} % math equations, symbols
\usepackage[english]{babel}
\usepackage{color}      % color content
\usepackage{graphicx}   % import figures
\usepackage{url}        % hyperlinks
\usepackage{bm}         % bold type for equations
\usepackage{multirow}
\usepackage{booktabs}
\usepackage{epstopdf}
\usepackage{epsfig}
\usepackage{algorithm}
\usepackage{algorithmic}

\usepackage[pagestyles]{titlesec}
% \renewcommand{\algorithmicrequire}{ \textbf{Input:}}     % use Input in the format of Algorithm  
% \renewcommand{\algorithmicinput}{ \textbf{Input:}}     % use Input in the format of Algorithm  
\renewcommand{\algorithmicensure}{ \textbf{Input:}} % use Initialize in the format of Algorithm  
% \renewcommand{\algorithmicreturn}{ \textbf{Output:}}     % use Output in the format of Algorithm  
\renewcommand{\figurename}{图}
% 引用参考文献标号显示在右上角
\newcommand{\upcite}[1]{\textsuperscript{\textsuperscript{\cite{#1}}}}

\newpagestyle{teststyle}{
  \sethead{\emph{fixed wings formation}}{\emph{\sectiontitle}}{\emph{\thepage \hspace{0.5em}page}}
  \renewcommand{\makeheadrule}{
    \makebox[0pt][l]{\rule[-.3\baselineskip]{\linewidth}{.5pt}}
    \rule[-.4\baselineskip]{\linewidth}{.5pt}
  }
}
\usepackage{color}
\usepackage{subfigure}
\usepackage{changepage}
\usepackage{fancyhdr} %设置页眉、页脚
\pagestyle{fancy}  %%%单线页眉
\fancyhead{}
\fancyhead[LO]{}
\fancyhead[RO]{}
% \fancyfoot[RO]{\thepage}
\fancypagestyle{plain}{%
  \pagestyle{fancy}
}
\usepackage{shorttoc}
\usepackage{xcolor}
\usepackage{mdframed}
\usepackage{titletoc}
% \renewcommand{\today}{\CJKnumber\year 年 \CJKnumber\month 月 \CJKnumber\day 日}

\DeclareRobustCommand{\chuhao}{\fontsize{42pt}{\baselineskip}\selectfont}  % 初号
\DeclareRobustCommand{\xiaochu}{\fontsize{36pt}{\baselineskip}\selectfont} % 小初
\DeclareRobustCommand{\yihao}{\fontsize{26pt}{\baselineskip}\selectfont}   % 一号
\DeclareRobustCommand{\xiaoyi}{\fontsize{24pt}{\baselineskip}\selectfont}  % 小一
\DeclareRobustCommand{\erhao}{\fontsize{22pt}{\baselineskip}\selectfont}   % 二号
\DeclareRobustCommand{\xiaoer}{\fontsize{18pt}{\baselineskip}\selectfont}  % 小二
\DeclareRobustCommand{\sanhao}{\fontsize{16pt}{\baselineskip}\selectfont}  % 三号 
\DeclareRobustCommand{\xiaosan}{\fontsize{15pt}{\baselineskip}\selectfont} % 小三
\DeclareRobustCommand{\sihao}{\fontsize{14pt}{\baselineskip}\selectfont}   % 四号
\DeclareRobustCommand{\xiaosi}{\fontsize{12pt}{\baselineskip}\selectfont}  % 小四
\DeclareRobustCommand{\wuhao}{\fontsize{10.5pt}{\baselineskip}\selectfont} % 五号
\DeclareRobustCommand{\xiaowu}{\fontsize{9pt}{\baselineskip}\selectfont}   % 小五
\DeclareRobustCommand{\liuhao}{\fontsize{7.5pt}{\baselineskip}\selectfont} % 六号
\DeclareRobustCommand{\xiaoliu}{\fontsize{6.5pt}{\baselineskip}\selectfont}% 小六
\DeclareRobustCommand{\qihao}{\fontsize{5.5pt}{\baselineskip}\selectfont}  % 七号

\lstset{numbers=left,numberstyle=\tiny,
breaklines=true,  %代码过长则换行
keywordstyle=\color{blue!70},commentstyle=\color{red!50!green!50!blue!50},frame=shadowbox, rulesepcolor=\color{gray!20!green!20!blue!20},escapeinside=``,xleftmargin=2em,xrightmargin=2em, aboveskip=1em}

\providecommand{\keywords}[1]{\textbf{\textit{keywords---}} #1}

 
\usepackage{hyperref} %bookmarks
% \usepackage[colorlinks,linkcolor=red,anchorcolor=blue,citecolor=green,CJKbookmarks=True]{hyperref}
\hypersetup{colorlinks, bookmarks, unicode} % unicode
 
\captionsetup[figure]{labelfont={bf},labelformat={default},labelsep=period,name={图}}
\newenvironment{figurehere}
{\def\@captype{figure}}
{}
 
% \title{\textbf{A*算法和模拟退火算法无人机领域的应用}}
% \author{ 冯学伟 \thanks{学号:2019520941}}
% \date{\today}

%标题、作者及日期
% \huge{\textbf{无人机导航控制}}\\[3mm]
% \Large{\textbf{The Navigational Control In The Field Of Unmanned Air Vehicle}}\\[1mm]
\title{
    \huge{\textbf{固定翼自主编队飞行}}\\
    \Large{\textbf{The Fixed-Wing Autonomous Formation Flying}}
}
\author{王曦漫, 吴昌伟, 冯学伟, 张效良}
\date{\today}

\begin{document}
    \maketitle   
    % \renewcommand{\contentsname}{Contents}  % 将Contents改为目录
    \tableofcontents
    % \thispagestyle{empty} % 设置当前页 页版式
    \clearpage % 分页/
   
    \renewcommand{\abstractname}{摘要}  % 将Abstract改为摘要
    \begin{center}
        \large{\textbf{摘要}}
    \end{center}
    \begin{adjustwidth}{0cm}{0cm}
        \hspace{2em} 无人飞行器(UnmannedAirVehicle.UAV)具有广阔的应用前景,是近年来高技术研究的热点目标之一。随着计算机技术,通信技术,传感器技术,电池技术等的
        飞速发展,开展微型UAV研究并把它运用到军事或民用中已经成为可能。
        \begin{flushleft}
        \par\textbf{关键字: } 无人机; 航线规划; PX4; %“\par在段首,表示另起一行,“\textbf{}”,花括号内的内容加粗显示
        \end{flushleft}
    \end{adjustwidth}
    \thispagestyle{empty} % 设置当前页 页版式
    \clearpage

    \begin{center}
        \large{\textbf{Abstract}}
    \end{center}
    \begin{adjustwidth}{0cm}{0cm}
        \hspace{1em} Unmanned Air Vehicle (Unmanned Air Vehicle.UAV) has broad application prospects and is one of the focuses of long-range high-tech research. Through computer technology, communication technology, sensor technology, battery technology, etc.
        Based on the PX4 control logic.
    \end{adjustwidth}
    \begin{keywords}
        \noindent Unmanned Air Vehicle; Path Manager; PX4     
    \end{keywords}
    \thispagestyle{empty} % 设置当前页 页版式
    \clearpage % 分页

    \setcounter{page}{1}        %从下面开始编页,页脚格式为导言部分设置的格式
    \pagestyle{teststyle}
    \section{绪论}
        引出本文话题
        \subsection{研究背景及意义}
            介绍研究背景等内容
        \subsection{国内外研究现状}
            \subsubsection{固定翼的研究现状}
            介绍研究现状之余, 也可以介绍一下多旋翼和固定翼的区别. 
            \subsubsection{固定翼自主编队飞行研究现状}
    \clearpage
    \section{Fixed Wing UAV Control Architecture}
            固定翼的体系结构(仿照px4官网), 一些飞行模式及其各个控制器介绍. 
            \subsection{Control Architecture Overview}
            Control Architecture Overview
                % 介绍一些飞行模式的执行步骤, 可以仿照px4官网, 比如有 takeoff, mission, landing, loiter, ALTCTL等.
                % \url{https://dev.px4.io/v1.9.0/en/concept/flight_modes.html}
            \subsection{navigator}
                控制器介绍
            \subsection{position controller}
                控制器介绍
            \subsection{attitude controller}
                控制器介绍
    \clearpage
    \section{UAVs Formation Algorithms}
        课题中主要用到的算法介绍.

        \subsection{Algorithms overview}
            算法伪代码
        \subsection{leader}
            主机用到的算法, 及其数据流走向
        \subsection{followers}
            从机的编队算法, 及其数据流走向

    \clearpage
    \section{Software In the Loop (SITL) simulation}
        介绍一些软件在环的
        \subsection{整体框图}
        \subsection{ROS}
            系统介绍, 及其使用机制
        \subsection{gazebo}
            系统介绍, 及其使用机制
        \subsection{QGround Control}
            系统介绍, 及其使用机制
        \subsection{Data Flow}
            上面部分串联起来, 描述软件在环数据流走向表示.
    \clearpage   
    \section{Hardware In the Loop (HITL) simulation}
    \subsection{整体框图}
        \subsection{飞控}
        系统简单介绍, 及其使用机制

        \subsection{树莓派}
            系统简单介绍, 及其使用机制

        \subsection{QGC}
            系统简单介绍, 及其使用机制

        \subsection{X-plane}
            系统简单介绍, 及其使用机制
    
        \subsection{Data Flow}
            上面部分串联起来, 描述硬件在环数据流走向表示.
    
    \clearpage    
    \section{Simulation}
        仿真结果
    \clearpage
    \section{Conclusion}
        总结
        \clearpage
    \section{reference}
        参考文献
\end{document}
