    \chapter{literature}
    \addcontentsline{toc}{section}{参考文献}
    \renewcommand{\refname}{参考文献}
    \begin{thebibliography}{99}
        \addtolength{\itemsep}{-2ex} % 用于更改行距
        \bibitem{1}杨玉,金敏,鲁华祥.融合简化稀疏A*算法与模拟退火算法的无人机航迹规划[J].计算机系统应用,2019,28(4):25-31. DOI:10.15888/j.cnki.csa.006864.
        \bibitem{2}张岳平,朱力超,孙涛.用Hopfield神经网络与模拟退火算法求解UAV航路规划问题[J].海军航空工程学院学报,2007,22(4):451-453,466. DOI:10.3969/j.issn.1673-1522.2007.04.012.
        \bibitem{3}赵梵喆,林跃,杨永琪.基于多目标规划的无人机路径规划[J].价值工程,2020,39(9):208-210.
        \bibitem{4}谭若晨.基于Multi-Agent系统的多UAV实时路径规划研究与实现[D].四川:电子科技大学,2013. DOI:10.7666/d.D772105.
        \bibitem{5}耿兴元.基于GPS与GIS的导航系统研究与开发[D].浙江:浙江大学,2004.
        \bibitem{6}张帅, 李学仁, 张鹏, 等.基于改进 A* 算法的无人机航迹规划[J] .飞行力学, 2016, 34( 3) : 39-43.
        \bibitem{7}Liu LF, Shi RX, Li SD, et al. Path planning for UAVS based
        on  improved  artificial  potential  field  method  through
        changing  the  repulsive  potential  function.  Proceedings  of
        2016  IEEE  Chinese  Guidance,  Navigation  and  Control
        Conference  (CGNCC).  Nanjing,  China.  2016.  2011–2015.
        \bibitem{8}D. R. Nelson, D. B. Barber, T. W. McLain and R. W. Beard, "Vector field path following for small unmanned air vehicles," 2006 American Control Conference, Minneapolis, MN, 2006, pp. 7 pp.-, doi: 10.1109/ACC.2006.1657648.
        \bibitem{9}Randal W. Beard and Timothy W. McLain, "Small Unmanned Aircraft: Theory and Practice", 2012, Princeton University Press
        \bibitem{10}R. W. {Beard} and T. W. {McLain} and D. B. {Nelson} and D. {Kingston} and D. {Johanson}, "Decentralized Cooperative Aerial Surveillance Using Fixed-Wing Miniature {UAVs}", 2006, Proceedings of the IEEE, 94, 7, 1306-1324
        \bibitem{11}R. W. {Beard} and J. {Ferrin} and J. {Humpherys}, "Fixed Wing {UAV} Path Following in Wind With Input Constraints", 2014, IEEE Transactions on Control Systems Technology, 2014, 22, 6, 2103-2117
        \bibitem{12}S. {Fari} and X. {Wang} and S. {Roy} and S. {Baldi}, "Addressing Unmodelled Path-Following Dynamics via Adaptive Vector Field: a {UAV} Test Case", 2019, IEEE Transactions on Aerospace and Electronic Systems, 10.1109/TAES.2019.2925487
    \end{thebibliography}  