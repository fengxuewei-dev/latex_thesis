\chapter{abstract}
\begin{adjustwidth}{0cm}{0cm}
    \emph{\textbf{摘\hspace{0.5em}要:} 以固定翼小型无人机的自主着陆控制为研究背景,提出了一种基于光流的固定翼小型无人机在移动降落架上的自主着
    陆控制方法。该方法首先\dots\dots; 其次\dots \dots; 最后在 simulink 环境下搭建动态仿真系统,仿真
    结果表明,使用本文方法可以有效实现飞行器的自主着陆控制. 
    % 在最后就机器学习在计算机视觉中的应用前景进行了展望。随机森林是近几年非常热门的分类方法,由于其训练和测试速度快,被许多学者用于图像匹配、动作识别等领域。利用在线的随机森林,能对跟踪目标进行不断学习,以获取目标最新的外观特征,从而及时完善跟踪,以达到最佳的状态。
    }
    \begin{flushleft}
    \emph{{\textbf{关键字:}} 固定翼小型无人机; 计算机视觉; 自主降落; 目标追踪; 光流}
    \end{flushleft}
\end{adjustwidth}
\begin{adjustwidth}{0cm}{0cm}
    \emph{\textbf{Abstract:}
    Taking the autonomous landing control of fixed-wing small unmanned aerial vehicles as the research background, 
    a method of autonomous landing control of fixed-wing small unmanned aerial vehicles based on optical flow is proposed. 
    This method, first of all, is \dots \dots, and \dots \dots, then uses the runway line as a feature to calculate its sparse linear optical flow field and combines The camera model and the relationship between the optical flow field and the velocity field use the horizontal flow of the runway line as system feedback to design the control system. Finally, a dynamic simulation system is built under the simulink environment. The simulation results show that the method of this paper can effectively achieve the autonomous landing control of the aircraft.}\par
\end{adjustwidth}
\begin{keywords}
    \emph{\noindent fixed-wing, Computer Vision, Autonomous landing, Target Tracking, light flow}
\end{keywords}