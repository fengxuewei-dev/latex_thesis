现在有了可以后台启动的方法, 如下所示. 
\begin{itemize}
    \item 1. 按照脚本的方法进行启动
    \item 2. 按照一条指令一条指令的执行(在rc.local文件中执行)
\end{itemize}
若要按照脚本的方法进行启动, 那么需要验证:
\begin{itemize}
    \item 当launch文件没有启动完全的时候, 后面的节点会等待(verified)
    \item 或者是设置一个小小的时延
\end{itemize}

脚本内部的执行逻辑
\begin{lstlisting}
#!/bin/bash
a=0
b=10
echo "Will it execute the source command!"
source ~/fixedWing_ws/devel/setup.bash
echo "Will it execute the UDP_Send command!"
rosrun communication UDP_Send 0 -mt UH -p 8002&
echo "Will it execute the while()!"
while [ "$a" != "$b" ];
do
#       /home/pi/UDP/UDP_Send &
        sleep 1s
        a=$(($a+1))
        echo "waiting 1s... $a"
done

echo "Will it open the px4.launch!"
roslaunch mavros px4.launch fcu_url:=/dev/serial0:921600 &
\end{lstlisting}


\textcolor{red}{在rc.local中执行脚本失败了, 卧槽!} 
现象是$UDP\_Send$是被执行了,
但是后面打开px4.launch文件失败了(会是while导致的吗?)

\textcolor{red}{执行输出打印在终端, 这个有点不可行}
中间的时延是必需的. 避免程序的并发性. 

\begin{itemize}
    \item 目的: 找到一个方法去让节点开机自启动; 
    \item 需要找到一个方法可以验证一个进程开机启动了; 
    \item 拿一个 UDP send 和 UDP Receive 来进行测试; 
    \item UDP send, UDP Receive 的本地源代码. 
    \item 验证方法, 修改rc.local文件是可以得到目的的
    \item 通过修改rc.local文件可以达到开机自启动一个脚本
    \item 有了开机自启动的方法之后, 启动一个脚本, 在脚本中添加执行逻辑
    \item 脚本内部的执行逻辑
\end{itemize}

在rc.local中执行单个指令是ok的, 但是不能执行脚本.\\ 
那如果执行两个指令呢? 是ok的.
执行ros相关指令没有通过

\href{https://blog.csdn.net/liuwinner/article/details/91040565}{解决rc.local不能执行脚本文件}

