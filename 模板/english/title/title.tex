\documentclass[UTF8, 11pt]{ctexart}

\usepackage[margin=1in]{geometry}
\usepackage{amsfonts,amsmath,amssymb}
\usepackage[none]{hyphenat}

% 宏包fancyhdr: 来设定文章的页眉页脚
\usepackage{fancyhdr}
\usepackage{graphicx}
\usepackage{float}

% 设置整体页面风格 \pagestyle{fancy}
% 也可以用 \thispagestyle{<风格>} 单独设置当前页的风格。book 类默认使用 heading 风格,report 和 article 默认使用 plain 风格
\pagestyle{fancy}

% 由于里面没有任何参数,所以这条命令用来清空所有的页眉设置。
\fancyhead{}
% 清除页脚设置
\fancyfoot{}

% E	偶数页
% O	奇数页
% L	左区域
% C	中间区域
% R	右区域	
% H	页眉	
% F	页脚
% \fancyhead[OR,EL]{\thepage} 表示页眉的奇数页右侧和偶数页左侧放置页码

% \slshape 本声明把字体的形状属性改为slanted的斜体,但保留族与系列不变
\fancyhead[L]{\slshape \MakeUppercase{Place Title Here}}
\fancyhead[R]{\slshape Student Name}
\fancyfoot[C]{\thepage}

% 定义一个新的命令来替换一段代码。
\renewcommand{\footrulewidth}{0pt}

%\parindent 0ex
% 这个命令会使该命令之后的所有段的缩进都变成这个值
\setlength{\parindent}{4em}
% 设置段落间距
\setlength{\parskip}{1em}
% 自己定义一个新的命令来替换一段代码。
\renewcommand{\baselinestretch}{1.5}

\begin{document}
	
\begin{titlepage}
    \begin{center} % 居中
        % 插入垂直间隔的命令是\vspace{长度}。
        \vspace*{1cm}
            % \textbf{内容} 将内容进行加粗
            \Large{\textbf{1B Mathematics SL}}\\
            \Large{\textbf{Internal Assessment}}\\
           
            \vfill % 用来具体的分割
           
            % \line(x-slope,y-slope){length}  一般是取 (1, 0)
            \line(1,0){400}\\[1mm] % create a horizontal line
            \huge{\textbf{This is a Sample Title}}\\[3mm]
            \Large{\textbf{-This is a SampleSubtitle-}}\\[1mm]
            \line(1,0){400}
           
            \vfill

            By Student Name\\
            Candidate \# \\
            \today \\
    \end{center}
\end{titlepage}

\tableofcontents
% empty: 无页眉页脚
% plain: 无页眉,页脚为居中页码
% headings: 页眉为章节标题,无页脚
% myheadings: 页眉内容可自定义,无页脚
\thispagestyle{empty} % 设置当前页 页版式
\clearpage % 分页

% 设置一个计数器 page
\setcounter{page}{1}
\section{introduction}

\section{Scoring Criteria}

\section{Communication}

    \subsection{Mathematical Presentation}

    \subsection{Personal Engagement}

    \subsection{Reflection}

    \subsection{Use of Mathematics}

\section{Conclusion}

\section{Using \LaTeX\ }

\end{document}
