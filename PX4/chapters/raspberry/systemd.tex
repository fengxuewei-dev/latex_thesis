\textcolor{red}{待测}\par
为了在树莓派启动时执行一个命令或程序,你可以设置一个服务。
一旦你有了一个服务,则可以使用start/stop/enable/disable来控制服务的执行. 
\subsection{创建服务}
在Pi上,你需要创建一个 .service 文件来创建服务。如下:
\begin{lstlisting}[title=$background_music.service$]
[Unit]
Description=BackgroundMusic
After=network.target

[Service]
ExecStart=/usr/bin/python3 -u play_audio.py
WorkingDirectory=/home/pi/myscript
StandardOutput=inherit
StandardError=inherit
Restart=always
User=pi

[Install]
WantedBy=multi-user.target
\end{lstlisting}
在这个示例中,服务会在使用python3来执行/home/pi/myscript目录下的 $play\_audio.py$ 脚本。
服务不仅限于执行python脚本,修改ExecStart后的命令即可执行程序或者shell命令。以root身份拷贝 \\
$background\_music.service$文件到etc/systemd/system目录下即可使用这个服务。

\subsection{启动/停止服务}
\begin{lstlisting}
    // start service
    sudo systemctl start background_music.service

    // stop service
    sudo systemctl stop background_music.service
\end{lstlisting}
\subsection{开机启动服务}
\begin{lstlisting}
    // 1. set to the auto start when start your machine
    sudo systemctl enable background_music.service

    // 2. cancel the auto start when start your machine
    sudo systemctl disable background_music.service
\end{lstlisting}
需要注意服务的启动依赖顺序: 
\begin{itemize}
    \item 服务需要在它所依赖的服务启动之后再启动。$background\_music.service$服务被指定在网络有效之后才启动(After=network.target)
    \item 服务的启动顺序和依赖可以在.service文件里配置
\end{itemize}
更多关于服务控制的细节,可以参考\textcolor{blue}{man systemctl}.