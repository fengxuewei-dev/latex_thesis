执行ls /etc/init.d 
可以看到该目录下有很多服务程序文件,
在这里添加自己的服务文件,
就可以对其进行配置从而实现自启动。
在该目录下新建文件 $Auto\_Start\_UDP\_Send$,编辑内容:
\begin{lstlisting}[title=服务文件配置]
#!/bin/bash
### BEGIN INIT INFO
# Provides: Auto_Start_Test
# Required-Start: $remote_fs
# Required-Stop: $remote_fs
# Default-Start: 2 3 4 5
# Default-Stop: 0 1 6
# Short-Description: Auto Start Test 
# Descrption: This service is used to test auto start service
### END INIT INFO

case "$1" in
    start)
        echo "Start"
        ./home/pi/UDP/UDP_Send &
        ;;
    stop)
        echo "Stop"
        killall ./home/pi/UDP/UDP_Send
        exit 1
        ;;
    *)
        echo "Usage:service Auto_Start_UDP_Send start|stop"
        exit 1
        ;;
esac
exit 0
\end{lstlisting}
\par 服务文件是什么: 
操作系统中的服务是指执行指定系统功能的程序、例程或进程,以便支持其他程序,
尤其是低层(接近硬件)程序。
通过网络提供服务时,服务可以在Active Directory(活动目录)中发布,
从而促进了以服务为中心的管理和使用。
\par 服务是一种应用程序类型,它在后台运行。
服务应用程序通常可以在本地和通过网络为用户提供一些功能,
例如客户端/服务器应用程序、Web服务器、
数据库服务器以及其他基于服务器的应用程序。(文件的权限r是4, w是2, x是1)
\par 按照下面的执行流程进行操作:
\begin{lstlisting}
    // 1. change the file's Authority 755 or 777
    sudo chmod 755 Auto_Start_UDP_Send

    // 2. add the service to Self-starting, than can be succeed
    sudo update-rc.d Auto_Start_UDP_Send defaults

    // 3. we can start the service by hand(had been verified).
    sudo service Auto_Start_Test start

    // 3. we can stop the service by hand(had been verified).
    sudo service Auto_Start_Test stop

    // 
    sudo reboot
\end{lstlisting}
之后, 重新启动树莓派查看效果. 不成功, 但是手动启动的时候是奏效的. 

那么如何将自己写的服务添加到开机自动启动呢? \par
\textcolor{blue}{chkconfig command}.如果没有该命令, 我们可以先进行安装
\begin{lstlisting}
    // install the chkconfig command
    sudo apt-get install chkconfig

    // 1. check out the services list and their priority
    chkconfig

    // 2. add the service to the list
    sudo chkconfig --add Auto_Start_UDP_Send 

    // 3. reboot(not verified)
    sudo reboot
\end{lstlisting}
\textcolor{red}{将启动服务指令放到rc.local文件中}\par
这里开启UDP$\_$Send, 需要网络服务
\begin{lstlisting}[title=查看服务的运行状态]
pi@raspberrypi:~ $ sudo service Auto_udp_send status
   Loaded: loaded (/etc/init.d/Auto_udp_send; generated; vendor preset: enabled)
   Active: active (exited) since Fri 2020-08-07 10:07:59 BST; 43s ago
     Docs: man:systemd-sysv-generator(8)
    Tasks: 0 (limit: 4915)
   CGroup: /system.slice/Auto_udp_send.service

Aug 07 10:07:59 raspberrypi Auto_udp_send[393]: Start
Aug 07 10:07:59 raspberrypi systemd[1]: Started LSB: Auto Start Test.
Aug 07 10:07:59 raspberrypi Auto_udp_send[393]: send heading    is:16
Aug 07 10:07:59 raspberrypi Auto_udp_send[393]: send airspeed   is:17
Aug 07 10:07:59 raspberrypi Auto_udp_send[393]: send position_x is:18
Aug 07 10:07:59 raspberrypi Auto_udp_send[393]: send position_y is:19
Aug 07 10:07:59 raspberrypi Auto_udp_send[393]: send position_z is:20
Aug 07 10:07:59 raspberrypi Auto_udp_send[393]: send yaw        is:21
Aug 07 10:07:59 raspberrypi Auto_udp_send[393]: message:1 Parity is:111
Aug 07 10:07:59 raspberrypi Auto_udp_send[393]: sendto error:: Network is unreachable
\end{lstlisting}
