\subsection{几种开机自启动的方法}
之前在树莓派上写的程序,都是通过ssh连接后在控制台上用命令行启动的,这种方式适合测试和调试,
完善好程序后,比较好的方法是把程序设置为开机自启动,
这样树莓派一上电就开始运行程序。查阅网上的资料,主要有三种方法,
% \url{https://blog.csdn.net/wangzhenyang2/article/details/80215149?utm_medium=distribute.pc_relevant.none-task-blog-BlogCommendFromMachineLearnPai2-1.nonecase&depth_1-utm_source=distribute.pc_relevant.none-task-blog-BlogCommendFromMachineLearnPai2-1.nonecase}
\par \href{https://blog.csdn.net/wangzhenyang2/article/details/80215149?utm_medium=distribute.pc_relevant.none-task-blog-BlogCommendFromMachineLearnPai2-1.nonecase&depth_1-utm_source=distribute.pc_relevant.none-task-blog-BlogCommendFromMachineLearnPai2-1.nonecase}{博客地址}
\begin{itemize}
    \item [1.] 是在$rc.local$添加启动项
    \item [2.] 是在$~/.config/autostart$中添加桌面启动应用
    \item [3.] 是在$/etc/init.d/$中添加服务项
    \item [4.] 使用systemctl设置服务
\end{itemize}
测试代码如下:
\begin{lstlisting}[title=UDP测试代码]
    #!/usr/bin/python
    import socket
    import time
    
    HOST='192.168.64.136'
    PORT=9999
    
    server=socket.socket(socket.AF_INET,socket.SOCK_DGRAM)
    server.connect((HOST,PORT))
    data='123'
    
    try:
        while True:
            server.sendall(data.encode(encoding='utf-8'))
            time.sleep(1)
    except:
        server.close()
\end{lstlisting}