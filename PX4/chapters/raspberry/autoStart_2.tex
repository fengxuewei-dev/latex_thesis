\subsection{几种开机自启动的方法}
之前在树莓派上写的程序,都是通过ssh连接后在控制台上用命令行启动的,这种方式适合测试和调试,
完善好程序后,比较好的方法是把程序设置为开机自启动,
这样树莓派一上电就开始运行程序。查阅网上的资料,主要有三种方法,
\par \href{https://blog.csdn.net/wangzhenyang2/article/details/80215149?utm_medium=distribute.pc_relevant.none-task-blog-BlogCommendFromMachineLearnPai2-1.nonecase&depth_1-utm_source=distribute.pc_relevant.none-task-blog-BlogCommendFromMachineLearnPai2-1.nonecase}{博客地址}
\begin{itemize}
    \item [1.] 是在$rc.local$添加启动项(已经实现)
    \item [2.] 是在$~/.config/autostart$中添加桌面启动应用
    \item [3.] 是在$/etc/init.d/$中添加服务项
    \item [4.] 使用systemctl设置服务
\end{itemize}
测试方法代码: \href{https://github.com/fengxuewei-dev/latex_thesis/blob/master/UDP/UDP_Send/UDP_Send}{$UDP\_Send$}
\subsubsection{通过rc.local自启动}
为了在树莓派启动的时候运行一个命令或程序,
你需要将命令添加到rc.local文件中。
这对于想要在树莓派接通电源后无需配置直接运行程序,
或者不希望每次都手动启动程序的情况非常有用。
\par 在你的树莓派上,选择一个文本编辑器编辑/etc/rc.local文件。你必须使用root权限编辑,例如
\\ $sudo\hspace{0.5em} vim\hspace{0.5em} /etc/rc.local$ 在注释后面添加命令, 但是要保证exit 0这行代码在最后, 然后保存文件退出. 
\textcolor{red}{如果你的命令需要长时间运行(例如死循环)
或者运行后不能退出,那么你必须确保在命令的最后添加$\&$符号让命令运行在其它进程, 
否则,这个脚本将无法结束,树莓派就无法启动。
这个$\&$符号允许命令运行在一个指定的进程中,
然后继续运行启动进程} \par
rc.local添加的执行指令, 会在启动的时候运行, 并且是在其他服务开启之前就启动了. 


\begin{lstlisting}[title=$/etc/rc.local$]
#!/bin/sh -e
#
# rc.local
#
# This script is executed at the end of each multiuser runlevel.
# Make sure that the script will "exit 0" on success or any other
# value on error.
#
# In order to enable or disable this script just change the execution
# bits.
#
# By default this script does nothing.

# Print the IP address
_IP=$(hostname -I) || true
if [ "$_IP" ]; then
  printf "My IP address is %s\n" "$_IP"
fi

./home/pi/UDP/UDP_Send &

exit 0
\end{lstlisting}

\par 执行sudo vim $/etc/rc.local$ 命令编辑rc.local文件, 
在exit 0上面一行添加语句\\$./home/pi/UDP/UDP\_Send\hspace{0.5em}\&$, 
表示运行$UDP\_Send$可执行文件,而$\&$符号可以简单理解为让程序运行在后台。
然后执行$sudo \hspace{0.5em} reboot$ 重启树莓派,记得把自己电脑上的接收程序提前打开,看看能不能接收到数据。
\par 启动之后, 可以在接收器上面执行接收到UDP数据.成功的时候如果输入$ps\hspace{0.5em}aux$ 来查看进程情况, 可以查看到
\begin{lstlisting}[title=进程执行]
USER       PID %CPU %MEM    VSZ   RSS TTY      STAT START   TIME COMMAND
root       541  0.7  0.0   3824   944 ?        S    08:38   0:00 ./home/pi/UDP/UDP_Send 
\end{lstlisting}

\textcolor{red}{Can't execute the scriptes by rc.local}\par
it's reported that not all programs will run reliably, 
because not all services may be available when rc.local runs.
\par 
we would have root ownership to execute availably all commands.




\subsubsection{通过桌面应用自启动}
    \textcolor{red}{待测}
\subsection{/etc/profile.d}
自己写一个shell脚本; 
将写好的脚本(.sh文件)放到目录 /etc/profile.d/ 下,
系统启动后就会自动执行该目录下的所有shell脚本。\par
将执行脚本放在 /etc/profile.d/ 路径下, 不可以执行, 
因为 /etc/profile.d/ 路径下的脚本文件都是在 /etc/profile 脚本启动的时候进行遍历执行的. 
\begin{lstlisting}[title=遍历执行/etc/profile.d/路径下的脚本文件]
if [ -d /etc/profile.d ]; then
  for i in /etc/profile.d/*.sh; do
    if [ -r $i ]; then
      . $i
    fi
  done
  unset i
fi

\end{lstlisting}
\textcolor{red}{待测}

\subsubsection{通过服务脚本自启动}
    执行ls /etc/init.d 
可以看到该目录下有很多服务程序文件,
在这里添加自己的服务文件,
就可以对其进行配置从而实现自启动。
在该目录下新建文件 $Auto\_Start\_UDP\_Send$,编辑内容:
\begin{lstlisting}[title=服务文件配置]
#!/bin/bash
### BEGIN INIT INFO
# Provides: Auto_Start_Test
# Required-Start: $remote_fs
# Required-Stop: $remote_fs
# Default-Start: 2 3 4 5
# Default-Stop: 0 1 6
# Short-Description: Auto Start Test 
# Descrption: This service is used to test auto start service
### END INIT INFO

case "$1" in
    start)
        echo "Start"
        ./home/pi/UDP/UDP_Send &
        ;;
    stop)
        echo "Stop"
        killall ./home/pi/UDP/UDP_Send
        exit 1
        ;;
    *)
        echo "Usage:service Auto_Start_UDP_Send start|stop"
        exit 1
        ;;
esac
exit 0
\end{lstlisting}
\par 服务文件是什么: 
操作系统中的服务是指执行指定系统功能的程序、例程或进程,以便支持其他程序,
尤其是低层(接近硬件)程序。
通过网络提供服务时,服务可以在Active Directory(活动目录)中发布,
从而促进了以服务为中心的管理和使用。
\par 服务是一种应用程序类型,它在后台运行。
服务应用程序通常可以在本地和通过网络为用户提供一些功能,
例如客户端/服务器应用程序、Web服务器、
数据库服务器以及其他基于服务器的应用程序。(文件的权限r是4, w是2, x是1)
\par 按照下面的执行流程进行操作:
\begin{lstlisting}
    // 1. change the file's Authority 755 or 777
    sudo chmod 755 Auto_Start_UDP_Send

    // 2. add the service to Self-starting, than can be succeed
    sudo update-rc.d Auto_Start_UDP_Send defaults

    // 3. we can start the service by hand(had been verified).
    sudo service Auto_Start_Test start

    // 3. we can stop the service by hand(had been verified).
    sudo service Auto_Start_Test stop

    // 
    sudo reboot
\end{lstlisting}
之后, 重新启动树莓派查看效果. 不成功, 但是手动启动的时候是奏效的. 

那么如何将自己写的服务添加到开机自动启动呢? \par
\textcolor{blue}{chkconfig command}.如果没有该命令, 我们可以先进行安装
\begin{lstlisting}
    // install the chkconfig command
    sudo apt-get install chkconfig

    // 1. check out the services list and their priority
    chkconfig

    // 2. add the service to the list
    sudo chkconfig --add Auto_Start_UDP_Send 

    // 3. reboot(not verified)
    sudo reboot
\end{lstlisting}
\textcolor{red}{将启动服务指令放到rc.local文件中}\par
这里开启UDP$\_$Send, 需要网络服务
\begin{lstlisting}[title=查看服务的运行状态]
pi@raspberrypi:~ $ sudo service Auto_udp_send status
   Loaded: loaded (/etc/init.d/Auto_udp_send; generated; vendor preset: enabled)
   Active: active (exited) since Fri 2020-08-07 10:07:59 BST; 43s ago
     Docs: man:systemd-sysv-generator(8)
    Tasks: 0 (limit: 4915)
   CGroup: /system.slice/Auto_udp_send.service

Aug 07 10:07:59 raspberrypi Auto_udp_send[393]: Start
Aug 07 10:07:59 raspberrypi systemd[1]: Started LSB: Auto Start Test.
Aug 07 10:07:59 raspberrypi Auto_udp_send[393]: send heading    is:16
Aug 07 10:07:59 raspberrypi Auto_udp_send[393]: send airspeed   is:17
Aug 07 10:07:59 raspberrypi Auto_udp_send[393]: send position_x is:18
Aug 07 10:07:59 raspberrypi Auto_udp_send[393]: send position_y is:19
Aug 07 10:07:59 raspberrypi Auto_udp_send[393]: send position_z is:20
Aug 07 10:07:59 raspberrypi Auto_udp_send[393]: send yaw        is:21
Aug 07 10:07:59 raspberrypi Auto_udp_send[393]: message:1 Parity is:111
Aug 07 10:07:59 raspberrypi Auto_udp_send[393]: sendto error:: Network is unreachable
\end{lstlisting}
    

\subsection{使用rc.local修改}
\begin{itemize}
    \item [1.] 在执行某一个语句之后, 等到几秒再次执行下一条语句
    \item [2.] 程序后台进行的脚本文件
\end{itemize}
\begin{lstlisting}[title=启动脚本多ROS节点]
#!/bin/bash
a=0
b=10
echo "Will it execute the source command!"
source ~/fixedWing_ws/devel/setup.bash
echo "Will it execute the UDP_Send command!"
rosrun communication UDP_Send 0 -mt UH -p 8002&
echo "Will it execute the while()!"
while [ "$a" != "$b" ];
do
#       /home/pi/UDP/UDP_Send &
        sleep 1s
        a=$(($a+1))
        echo "waiting 1s... $a"
done

echo "Will it open the px4.launch!"
roslaunch mavros px4.launch fcu_url:=/dev/serial0:921600 &
\end{lstlisting}
如果$\#!/bin/sh$, 会出现$source:not found!$, 解决办法是修改为$\#!/bin/bash$

\textcolor{red}{在rc.local中执行脚本失败了, 卧槽!} 
现象是$UDP\_Send$是被执行了,
但是后面打开px4.launch文件失败了(会是while导致的吗?)

\subsection{init.d目录和/etc/rc.local的区别}

init.d目录中的脚本都是以服务的形式启动的,顾名思义,服务会在后台一直运行. 
所以,系统在执行init.d目录中的服务脚本时,会分别单独为每个服务脚本启动一个
非登录非交互式shell来始终在后台运行服务脚本一直到用户退出登录,关闭系统,
这些始终运行在各个非登录非交互式的shell中的服务脚本才会停止运行 \par
rc.local也是我经常使用的一个脚本。该脚本是在系统初始化级别脚本运行之后再执行的,因此可以安全地在里面添加你想在系统启动之后执行的脚本, 启动顺序可见\ref{rasp:startup}。\par
可以在rc.local中启动一些服务. 

\subsection{查看服务优先顺序}
要知道服务的启动顺序,
就需要先知道服务如何启动的。
linux有7个运行级别
,用户可选择不同的运行级别。
进入/etc/rc.d/目录,
可查看到对应从rc0.d到rc6.d等7个目录,
这些目录即对应7个级别。
\begin{lstlisting}[title=查看运行级别]
    pi@raspberrypi:/etc $ who -r 
         run-level 5  2020-08-07 11:28
\end{lstlisting}
查看自己系统的运行级别指令为\textcolor{red}{who -r}. 进入rc3.d目录,
可看到各种以K或者S开始的服务,
命名都以S(start)、K(kill)或D(disable)开头, 
而后面的数字就表示启动顺序。
我们以熟悉的network服务为例,
\textcolor{red}{这里只是个链接},其实还是指向/etc/init.d/network,其启动值为10。
