为了在树莓派启动的时候运行一个命令或程序,
你需要将命令添加到rc.local文件中。
这对于想要在树莓派接通电源后无需配置直接运行程序,
或者不希望每次都手动启动程序的情况非常有用。
\par 在你的树莓派上,选择一个文本编辑器编辑/etc/rc.local文件。你必须使用root权限编辑,例如
\\ $sudo\hspace{0.5em} vim\hspace{0.5em} /etc/rc.local$ 在注释后面添加命令, 但是要保证exit 0这行代码在最后, 然后保存文件退出. 
\textcolor{red}{如果你的命令需要长时间运行(例如死循环)
或者运行后不能退出,那么你必须确保在命令的最后添加$\&$符号让命令运行在其它进程, 
否则,这个脚本将无法结束,树莓派就无法启动。
这个$\&$符号允许命令运行在一个指定的进程中,
然后继续运行启动进程} \par
rc.local添加的执行指令, 会在启动的时候运行, 并且是在其他服务开启之前就启动了. 

