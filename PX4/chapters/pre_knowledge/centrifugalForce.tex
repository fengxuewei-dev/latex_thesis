当物体在做非直线运动时(非牛顿环境,例如:圆周运动或转弯运动),
因物体一定有本身的质量存在,
质量造成的惯性会强迫物体继续朝着运动轨迹的切线方向(原来那一瞬间前进的直线方向)前进,
而非顺着接下来转弯过去的方向走。
\par 若这个在做非直线运动的物体(例如:车)上有乘客的话,乘客由于同样随着车子做转弯运动,
会受到车子向乘客提供的向心力,但是若以乘客为参照系,由于该参照系为非惯性系,
他会受到与他相对静止的车子给他的一个指向圆心的向心力作用,
但同时他也会给车子一个反向等大,由圆心指向外的力,就好像没有车子他就要被甩出去一样,这个力就是所谓的离心力。
\clearpage
