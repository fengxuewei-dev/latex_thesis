\textcolor{blue}{shell编程中的命令有时和C语言是一样的。$\&\&$表示与,$\|\|$表示或。把两个命令用$\&\&$联接起来,
如 make mrproper $\&\&$ make menuconfig,表示要第一个命令执行成功才能执行第二个命令。对执行顺序有要求的命令能保证一旦有错误发生,下面的命令不会盲目地继续执行}.
\subsection{source and ./}
    source命令:\\
    source命令也称为“点命令”,也就是一个点符号(.),是bash的内部命令。\\
    功能:使Shell读入指定的Shell程序文件并依次执行文件中的所有语句\\
    source命令通常用于重新执行刚修改的初始化文件,使之立即生效,而不必注销并重新登录。\\
    用法:source filename 或 . filename
\par source filename:这个命令其实只是简单地读取脚本里面的语句
依次在当前shell里面执行,没有建立新的子shell
。那么脚本里面所有新建、
改变变量的语句都会保存在当前shell里面。
\par \textcolor{blue}{它的作用就是把一个文件的内容当成是shell来执行}
\subsection{bash}
\subsection{sh}

sh filename 重新建立一个子shell,
在子shell中执行脚本里面的语句,
该子shell继承父shell的环境变量,
但子shell新建的、
改变的变量不会被带回父shell,除非使用export。
\par
当shell脚本具有可执行权限时,
用sh filename与./filename执行脚本是没有区别得。
./filename是因为当前目录没有在PATH中,
所有"."是用来表示当前目录的。
\subsection{区别}
\begin{itemize}
    \item 
\end{itemize}