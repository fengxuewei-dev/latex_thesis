\subsection{nohup}
用nohup运行命令可以使命令永久的执行下去,和用户终端没有关系,
例如我们断开SSH连接都不会影响他的运行,注意了nohup没有后台运行的意思;$\&$才是后台运行 \par
$\&$是指在后台运行,但当用户推出(挂起)的时候,命令自动也跟着退出. 

\begin{itemize}
    \item 使用$\&$后台运行程序:
    \item 结果会输出到终端
    \item 使用Ctrl + C发送SIGINT信号,程序免疫
    \item 关闭session发送SIGHUP信号,程序关闭
\end{itemize}
\begin{itemize}
    \item 使用nohup运行程序:
    \item 结果会输出到终端
    \item 使用Ctrl + C发送SIGINT信号,程序免疫
    \item 关闭session发送SIGHUP信号,程序关闭
\end{itemize}
平日, 经常使用nohup和$\&$配合来启动程序: 同时免疫SIGINT和SIGHUP信号.\par
同时,还有一个最佳实践:不要将信息输出到终端标准输出,标准错误输出,而要用日志组件将信息记录到日志里

