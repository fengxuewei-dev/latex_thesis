在刚登录Linux时,
首先启动 /etc/profile 文件,
然后再启动用户目录下的$ ~/.bash\_profile$、
$~/.bash\_login$或 
~/.profile文件中的其中一个,
执行的顺序为:$ ~/.bash\_profile$、$~/.bash\_login$、 ~/.profile。(profile: 配置文件)



\subsection{各个配置文件的作用域}
\textcolor{blue}{$~/.bash\_profile$: 是交互式、login 方式进入 bash 运行的} \par
\textcolor{blue}{~/.bashrc 是交互式 non-login 方式进入 bash 运行的. 通常二者设置大致相同,所以通常前者会调用后者}.
\subsubsection{$/etc/profile$}
此文件为系统的每个用户设置环境信息,当用户第一次登录时,该文件被执行. 并从/etc/profile.d目录的配置文件中搜集shell的设置。

\subsubsection{$/etc/bashrc$}
为每一个运行bash shell的用户执行此文件.当bash shell被打开时,该文件被读取(即每次新开一个终端,都会执行bashrc)。

\subsubsection{$~/.bash\_profile$}
每个用户都可使用该文件输入专用于自己使用的shell信息,当用户登录时,该文件仅仅执行一次。默认情况下,设置一些环境变量,执行用户的.bashrc文件。

\subsubsection{$~/.bashrc$}
该文件包含专用于你的bash shell的bash信息,当登录时以及每次打开新的shell时,该该文件被读取。

\subsubsection{$~/.bash\_logout$}
当每次退出系统(退出bash shell)时,执行该文件. 另外,/etc/profile中设定的变量(全局)的可以作用于任何用户,而~/.bashrc等中设定的变量(局部)只能继承 /etc/profile中的变量,他们是"父子"关系。



