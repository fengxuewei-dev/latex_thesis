\subsection{raspberry's handled logic and px4's receiver}
树莓派的控制指令以mavlink消息$(MAVLINK_MSG_ID_RC_CHANNELS_OVERRIDE = 70)$进入到px4内部, 处理函数名字为:
\begin{lstlisting}[title=mavlink处理函数声明体]
    // Firmware/src/modules/mavlink/mavlink_receiver.cpp
    void MavlinkReceiver::handle_message(mavlink_message_t *msg);
\end{lstlisting}
\par 处理函数内部会对信息进行解码处理,得到18个通道(channels)的值,并且赋值给 $input\_rc\_s$ rc{}变量, 之后进行
有效性的处理(判断值18个通道的值是否为65535或者是0, 若是, 则该通道的值变为0; 反之, 该通道的值不变,且更新$rc.channel\_count$的值),
处理之后, 使用发布器 $\_rc\_pub$ 将该变量发布到 \\ $ORB\_ID(input\_rc)$话题上(定义发布器的时候, 会设置到PublicationMulti机制, 实例化会有一个优先级的赋值, 可以参见\textcolor{red}{需要补充} )
树莓派发送的四个值占用前四个通道,这个四个通道对应的顺序和遥控器对应指令的通道是不一样的 $(roll = 1 , pitch = 2, throttle = 3, yaw = 4)$.
\begin{lstlisting}[title=$input\_rc\_s$结构体的定义]
    struct input_rc_s {
        uint64_t timestamp;
        uint64_t timestamp_last_signal;
        uint32_t channel_count;
        int32_t rssi;
        uint16_t rc_lost_frame_count;
        uint16_t rc_total_frame_count;
        uint16_t rc_ppm_frame_length;
        uint16_t values[18];
        bool rc_failsafe;
        bool rc_lost;
        uint8_t input_source;
        uint8_t _padding0[3]; // required for logger
    }
    // publisher declaration 
    uORB::PublicationMulti<input_rc_s>			_rc_pub{ORB_ID(input_rc), ORB_PRIO_LOW};
\end{lstlisting}
\begin{lstlisting}[title=树莓派以及固件通道定义]
    // raspberry
    #define PITCH_CHANNEL 		1
    #define ROLL_CHANNEL 		2
    #define YAW_CHANNEL 		3
    #define THROTTLE_CHANNEL 	4
    
    // px4
    #define PITCH_CHANNEL 		1
    #define ROLL_CHANNEL 		2
    #define YAW_CHANNEL 		3
    #define THROTTLE_CHANNEL 	4
\end{lstlisting}

\subsection{其他部分的订阅}

mavlink接收函数拿到数据之后,进行一些处理之后, 将数据publish出去。在$rc\_update$中进行订阅
(multi publish 及其 subscribe 机制在后续文章中进行讲解),订阅器定义如下:
\begin{lstlisting}[title=订阅器的声明定义]
    // src/modules/sensors/rc_update.h
    uORB::Subscription	_rc_sub{ORB_ID(input_rc)};				/**< raw rc channels data subscription */
\end{lstlisting}

在源文件中$rc\_poll\ api$中进行更新获取该变量的值,再进行一些逻辑标志位的检索,
最后对其进行第二次的有效性检测(将每个通道的值约束到某一个固定的区间, 类似constrained()函数; trim操作, 
准备数据为manual id publish做准备), 再一次更新元素的值. 更新后的值处理.
\begin{itemize}
    \item [(1)] 发布到 ORB\_ID(rc\_channels) 话题上面.
    \item [(2)] 将 struct manual\_control\_setpoint\_s manual = {} 数据进行约束到[-1.0f, 1.0f]之间更新到该变量中,继而 发布到 ORB\_ID(manual\_control\_setpoint) 消息上.
    \item [(3)] 将第二次发布得到的 manual 变量, 赋值给 actuator\_group\_3.control 数组, 发布到\\ ORB\_ID(actuator\_controls\_3)消息上.
\end{itemize}

我们真正关心的消息topic应该是$(1\>, 2\>)$, 也就是 $ORB\_ID(rc\_channels)$ 和 $ORB\_ID(manual\_control\_setpoint)$, 
其中 $ORB\_ID(rc\_channels)$ 会被作用到PWM(遥控器和接收机的通信方式)上,所以,我们只需要关心的就是\\ $ORB\_ID(manual\_control\_setpoint)$ 消息topic即可.
\par
上述变量及其函数定义如下:
\begin{lstlisting}[title=上述变量及其函数定义]
    void RCUpdate::rc_poll(const ParameterHandles &parameter_handles);


    rc_channels_s _rc {};			/**< r/c channel data */
    orb_publish_auto(ORB_ID(rc_channels), &_rc_pub, &_rc, &instance, ORB_PRIO_DEFAULT);
    
    
    struct manual_control_setpoint_s manual = {};
    orb_publish_auto(ORB_ID(manual_control_setpoint), &_manual_control_pub, &manual, &instance,
                ORB_PRIO_HIGH);
    
    
    /* copy from mapped manual control to control group 3 */
    struct actuator_controls_s actuator_group_3 = {};
    /* publish actuator_controls_3 topic */
    orb_publish_auto(ORB_ID(actuator_controls_3), &_actuator_group_3_pub, &actuator_group_3, &instance,
                 ORB_PRIO_DEFAULT);
\end{lstlisting}

\subsection{ORB\_ID(manual\_control\_setpoint)}
这个uORB消息, 在px4内部会被 FixedWing Position Controller , FixedWing Attitude Controller 及其他原件进行订阅使用, 
这里我们需要关心的 FixedWing Position Controller , FixedWing Attitude Controller中的使用情况.
\subsubsection{FixedWing Position Controller}
在制导控制器中, px4会根据当前的throttle期望值, 调用 内部的 TECS, 进行新的姿态设定值, 计算期望空速, pitch, 以及使用其他的逻辑来进行计算 yaw, 以及roll的设定值, 赋值给变量$\_att\_sp$, 从而在最后发布给下一层的姿态控制器. 
\begin{lstlisting}[title=计算一些姿态的设定值]
_att_sp.roll_body = _manual.y * _parameters.man_roll_max_rad;
_att_sp.yaw_body = 0;

const float deadBand = 0.06f;
float factor = 1.0f - deadBand;
float pitch = -(_manual.x + deadBand) / factor;

// calculate the demanded airspeed.
float
FixedwingPositionControl::get_demanded_airspeed()
{
	float altctrl_airspeed = 0; // the demanded airspeed.

	// neutral throttle corresponds to trim airspeed
	if (_manual.z < 0.5f) {
		// lower half of throttle is min to trim airspeed
		altctrl_airspeed = _parameters.airspeed_min +
				   (_parameters.airspeed_trim - _parameters.airspeed_min) *
				   _manual.z * 2;

	} else {
		// upper half of throttle is trim to max airspeed
		altctrl_airspeed = _parameters.airspeed_trim +
				   (_parameters.airspeed_max - _parameters.airspeed_trim) *
				   (_manual.z * 2 - 1);
	}

	return altctrl_airspeed;
}
\end{lstlisting}

\textcolor{red}{上述代码公式转换如下}

\subsubsection{FixedWing Attitude Controller}
姿态控制器拿到数据且赋值给 $\_manual$ 变量.
\begin{lstlisting}[title=高度处理逻辑]
if (_vcontrol_mode.flag_control_rattitude_enabled) {
	if (fabsf(_manual.y) > _parameters.rattitude_thres || fabsf(_manual.x) > _parameters.rattitude_thres) {
		_vcontrol_mode.flag_control_attitude_enabled = false;
	}
}
\end{lstlisting}
\par
\begin{itemize}
    \item 若该变量的y和x大于 $\_parameters.rattitude\_thres$ 参数的值, 则 $flag\_control\_attitude\_enabled = false$, 若这个时候$flag\_control\_rates\_enabled$ 为真, 那么执行处理逻辑1; 再将值发布到\\$ORB\_ID(vehicle\_rates\_setpoint)$消息上, 进行下一步的处理.
    \item 若该变量的y和x小于等于 $\_parameters.rattitude\_thres$ 参数的值, 则 $flag\_control\_attitude\_enabled = true$, 执行处理逻辑2, 将值publish到 $\_attitude\_setpoint\_id$ 上面(这个topic就对应offboard从mavros发送到px4的控制逻辑层)
    \item 若不符合上面两个逻辑, 直接执行处理逻辑3, 将控制指令直接发布给执行器. 将值publish到 \\ $\_attitude\_setpoint\_id$ 上
\end{itemize}
\begin{lstlisting}[title=处理逻辑1]  
    _rates_sp.roll = _manual.y * _parameters.acro_max_x_rate_rad;
    _rates_sp.pitch = -_manual.x * _parameters.acro_max_y_rate_rad;
    _rates_sp.yaw = _manual.r * _parameters.acro_max_z_rate_rad;
    _rates_sp.thrust_body[0] = _manual.z;        
\end{lstlisting}
\begin{lstlisting}[title=处理逻辑2]  
    // STABILIZED mode generate the attitude setpoint from manual user inputs
					_att_sp.timestamp = hrt_absolute_time();

					// calculate the setpoints 
					_att_sp.roll_body = _manual.y * _parameters.man_roll_max + _parameters.rollsp_offset_rad;
					_att_sp.roll_body = math::constrain(_att_sp.roll_body, -_parameters.man_roll_max, _parameters.man_roll_max);
					_att_sp.pitch_body = -_manual.x * _parameters.man_pitch_max + _parameters.pitchsp_offset_rad;
					_att_sp.pitch_body = math::constrain(_att_sp.pitch_body, -_parameters.man_pitch_max, _parameters.man_pitch_max);
					_att_sp.yaw_body = 0.0f;
					_att_sp.thrust_body[0] = _manual.z;

					// get the Quatf
					Quatf q(Eulerf(_att_sp.roll_body, _att_sp.pitch_body, _att_sp.yaw_body));
					q.copyTo(_att_sp.q_d);
                    _att_sp.q_d_valid = true;  
                    
                    _attitude_setpoint_id = ORB_ID(vehicle_attitude_setpoint);
\end{lstlisting}
\begin{lstlisting}[title=处理逻辑3]  	
    /* manual/direct control */
    _actuators.control[actuator_controls_s::INDEX_ROLL] = _manual.y * _parameters.man_roll_scale + _parameters.trim_roll;
    _actuators.control[actuator_controls_s::INDEX_PITCH] = -_manual.x * _parameters.man_pitch_scale + _parameters.trim_pitch;
    _actuators.control[actuator_controls_s::INDEX_YAW] = _manual.r * _parameters.man_yaw_scale + _parameters.trim_yaw;
    _actuators.control[actuator_controls_s::INDEX_THROTTLE] = _manual.z;

    _actuators_id = ORB_ID(actuator_controls_0);
\end{lstlisting}

\begin{lstlisting}[title=$\_attitude\_setpoint\_id$]
_attitude_setpoint_id = ORB_ID(vehicle_attitude_setpoint);
\end{lstlisting}

\subsection{Low pass filter}
\begin{lstlisting}[title=Low pass filter from px4]
    float LowPassFilter2p::apply(float sample)
{
	// do the filtering
	float delay_element_0 = sample - _delay_element_1 * _a1 - _delay_element_2 * _a2;

	if (!PX4_ISFINITE(delay_element_0)) {
		// don't allow bad values to propagate via the filter
		delay_element_0 = sample;
	}

	const float output = delay_element_0 * _b0 + _delay_element_1 * _b1 + _delay_element_2 * _b2;

	_delay_element_2 = _delay_element_1;
	_delay_element_1 = delay_element_0;

	// return the value. Should be no need to check limits
	return output;
}
\end{lstlisting}
    